\PassOptionsToPackage{unicode=true}{hyperref} % options for packages loaded elsewhere
\PassOptionsToPackage{hyphens}{url}
%
\documentclass[]{ltjarticle}
\usepackage[top=30truemm,bottom=30truemm,left=25truemm,right=20truemm]{geometry}
\usepackage{lmodern}
\usepackage{amssymb,amsmath}
\usepackage{ifxetex,ifluatex}
\usepackage{fixltx2e} % provides \textsubscript
\ifnum 0\ifxetex 1\fi\ifluatex 1\fi=0 % if pdftex
  \usepackage[T1]{fontenc}
  \usepackage[utf8]{inputenc}
  \usepackage{textcomp} % provides euro and other symbols
\else % if luatex or xelatex
  \usepackage{unicode-math}
  \defaultfontfeatures{Ligatures=TeX,Scale=MatchLowercase}
\fi
% use upquote if available, for straight quotes in verbatim environments
\IfFileExists{upquote.sty}{\usepackage{upquote}}{}
% use microtype if available
\IfFileExists{microtype.sty}{%
\usepackage[]{microtype}
\UseMicrotypeSet[protrusion]{basicmath} % disable protrusion for tt fonts
}{}
\IfFileExists{parskip.sty}{%
\usepackage{parskip}
}{% else
\setlength{\parindent}{0pt}
\setlength{\parskip}{6pt plus 2pt minus 1pt}
}
\usepackage[colorlinks,linkcolor=blue,urlcolor=blue]{hyperref}
\hypersetup{
            pdfborder={0 0 0},
            breaklinks=true}
\urlstyle{same}  % don't use monospace font for urls
\setlength{\emergencystretch}{3em}  % prevent overfull lines
\providecommand{\tightlist}{%
  \setlength{\itemsep}{0pt}\setlength{\parskip}{0pt}}
\setcounter{secnumdepth}{5}
% Redefines (sub)paragraphs to behave more like sections
\ifx\paragraph\undefined\else
\let\oldparagraph\paragraph
\renewcommand{\paragraph}[1]{\oldparagraph{#1}\mbox{}}
\fi
\ifx\subparagraph\undefined\else
\let\oldsubparagraph\subparagraph
\renewcommand{\subparagraph}[1]{\oldsubparagraph{#1}\mbox{}}
\fi

% set default figure placement to htbp
\makeatletter
\def\fps@figure{htbp}
\makeatother


\usepackage{fancyhdr}
\pagestyle{fancy}
\lhead{Ubuntuのバージョンアップ}



\begin{document}

\tableofcontents \newpage

\title{Ubuntuのバージョンアップ}
%\author{\leftline{{\small{廣野秀樹}}}}
\author{廣野秀樹}
\date{2020.05.05}
\maketitle




\hypertarget{ubuntuux306eux30d0ux30fcux30b8ux30e7ux30f3ux30a2ux30c3ux30d7}{%
\section{Ubuntuのバージョンアップ}\label{ubuntuux306eux30d0ux30fcux30b8ux30e7ux30f3ux30a2ux30c3ux30d7}}

\hypertarget{ubuntu19.10ux304bux308920.04ux3078ux306eux30a2ux30c3ux30d7ux30b0ux30ecux30fcux30c9ux306820.04ux306eux65b0ux898fux30a4ux30f3ux30b9ux30c8ux30fcux30ebux4f5cux696d}{%
\subsection{Ubuntu19.10から20.04へのアップグレードと,20.04の新規インストール作業}\label{ubuntu19.10ux304bux308920.04ux3078ux306eux30a2ux30c3ux30d7ux30b0ux30ecux30fcux30c9ux306820.04ux306eux65b0ux898fux30a4ux30f3ux30b9ux30c8ux30fcux30ebux4f5cux696d}}

:CATEGORIES: Linux,Ubuntu

 アップグレードは夕方に始めたように思います。まずはじめにデータのバックアップ以外にほとんど起動することのないディスクトップの古いパソコンにインストールする作業を午後に始めました。しかし,DVDからインストールすることが出来ず,アップグレードを試みました。

 インストールのISOファイルをDVDに書き込むことが出来ず,2枚のDVD-Rをだめにして,夕方に買い物のついでに宇出津新港のアルプの百均で2枚組のDVD-Rと1枚のDVD-RWを買ってきたのですが,これは外付けのプレーヤーに入れても認識をしませんでした。

 買い物から戻ったのが18時を過ぎていて,買い物に出かける前にUSBにインストールディスクを作成していたのですが,古いディスクトップパソコンの方はUSBから起動することが出来ませんでした。数年前,本体のみをタダでもらったパソコンになります。

 もともとデータのバックアップ用にサーバー専用機のような用途をしていたので,ディスクトップのアプリや込み入ったライブラリー余り入れていないので,アップグレードが出来るのではないかと思い調べてみました。

 Ubuntuのシステムのアップグレードは10年以上前,羽咋市に住んでいた頃に2,3度やったことがあったのですが,ライブラリーのバージョンの違いなどで正常に動くことがなく,エラーの対処も多かったので,新規インストール以外はしなくなっていました。

 百均に売っているDVD-Rが古いDVDプレーヤーでは使えない時代となってしまいました。新しい外付けのDVDプレーヤーを買うという方法もありますが,OSのインストールではシステムの最初の起動時にハードディスクやSSD以外からBOOTが出来なければなりません。

 普通に考えて,USBからブートできるのであれば,USB接続のDVDプレーヤーからブートできる可能性はありそうですが,USBからブートできないことを確認したパソコンになるので,できる見込みはありません。

 アップグレードが成功した時間のことは憶えていないですが,昨日の午後から夕方遅くに掛け,4,5時間はその作業に時間を使ったように思います。アップグレードではMySQLのインストールや設定がうまくいかず,それにもかなり時間を掛けていました。

 あとでUbuntu20.04の新規インストールをした時は,余り問題が起きなかったのですが,アップグレードの場合は,ネットで調べながらけっこう面倒な作業を行いました。MySQLのバージョンが上がったためで,他にもコマンドの書式に変更点があったようです。

» Ubuntu 20.04 MySQL - Google 検索 https://t.co/gVGMo1OIIF

» Ubuntu 20.04にMySQL 8.0をインストール(OS標準) - Qiita
https://t.co/SdV1RDvlks

» Linuxディストリビューション「Ubuntu 20.04
LTS」リリース、2年ぶりの長期サポート版 - クラウド Watch
https://t.co/gQEmt3K3mw

 あらためて調べても,MySQLのバージョンアップは情報が乏しく関心も少ないようですが,バージョンが8になって処理速度が2倍になったという話がありました。確かに処理速度は早くなったような実感があります。

 pythonのデフォルトが2系から3系になるという話は前に見かけていたのですが,MySQLのバージョンアップは情報を見かけることがなく,今調べてもごくわずかな情報です。データベースの移行もうまくはいったのですが,すんなりとはいかず,色々やっていました。

 今回,Ubuntu20.04の新規インストールをしたのはメインで使っているディスクトップパソコンですが,19.04で使っていたパーティションを初期化するかたちでインストールしました。昨日までメインで使っていた18.04は,そのまま残っています。

 18.04をインストールしていたのは2.5インチの100GBほどのSSDで,値段も3千円より安かったかもしれません。Amazonで購入しました。移行した20.04のUbuntuに問題がなければ,ノートパソコンに取り付け,そちらにも20.04の新規インストールを考えています。

 ここ数年,Ubuntuのバージョンアップにトラブルは少なくなっているのですが,今回も思わぬところで不具合が出たり対処を迫られてました。のんびりやっていたのですが,午前4時半ぐらいまで,その後の設定作業などをやっていました。

 今朝になって気がついたのですが,Emacsでも多少の不具合があって,その1つがフォントの大きさになります。フォントの設定はネットで探してもいくつかの方法があるのですが,設定が通用する場合としない場合があって,今回もやや不満を残したままの状態となっています。

 単純に日本語のフォントサイズが大きくなったのですが,設定で改善はみられるものの,なぜか空行の幅が大きいのです。なぜか,フォントを変更するメニューも出てきません。Emacsのインストールが不完全なのかもしれません。

 Ubuntuのバージョンアップは一年に2回あって,4月と10月になります。今回は2020年4月を意味する20.04というバージョンです。今回はLTSということでサポート期間の長いバージョンです。

 19.10もインストールした環境があったのですが,19.04ではサポート切れでアップデートが出来ない状態となっていました。長年,Ubuntuを使ってきましたが,これは19.04が初めてのことでした。

 以前は,半年に1回,Ubuntuの新規インストールを決まって行っていたのですが,昨年あたりから18.04のLTSをそのまま使い続けるようにもなっていました。なにかとトラブルも起こり,時間と手間の掛かる作業になりますが,繰り返し学ぶ,勉強の機会ともなります。

 UbuntuなどのLinuxの場合,Windowsパソコンのようにインストールディスクに使うということはなく,インターネット接続の環境さえあれば,いつでもインストールや更新が出来る仕組みとなっています。

\end{document}


